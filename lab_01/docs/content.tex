\chapter{Задание}

\textbf{Цель работы:} построение гистограммы и эмпирической функции распределения.


Для выборки объёма $n$ из генеральной совокупности $X$ реализовать в виде программы на ЭВМ
        \begin{enumerate}\begin{enumerate}
            \item вычисление максимального значения $M_{\max}$ и минимального значения $M_{\min}$;
            \item размаха $R$ выборки;
            \item вычисление оценок $\hat\mu$ и $S^2$ математического ожидания $MX$ и дисперсии $DX$;
            \item группировку значений выборки в $m = [\log_2 n] + 2$ интервала;
            \item построение на одной координатной плоскости гистограммы и графика функции плотности распределения вероятностей нормальной случайной величины с математическим ожиданием $\hat{\mu}$ и дисперсией $S^2$;
            \item построение на другой координатной плоскости графика эмпирической функции распределения и функции распределения нормальной случайной величины с математическим ожиданием $\hat{\mu}$ и дисперсией $S^2$.
        \end{enumerate}\end{enumerate}
Провести вычисления и построить графики для выборки из индивидуального варианта.

%\textbf{Выборка индивидуального варианта №8:}

%\begin{lstlisting}
%X=(7.76,6.34,5.11,7.62,8.84,4.68,8.65,6.90,8.79,6.61,6.62,7.13,6.75,7.28,7.74,7.08,5.57,8.20,7.78,7.92,6.00,4.88,6.75,6.56,7.48,8.51,9.06,6.94,6.93,7.79,5.71,5.93,6.81,5.76,5.88,7.05,7.22,6.67,5.59,6.57,7.28,6.22,6.31,5.51,6.69,7.12,7.40,6.86,7.28,6.82,7.08,7.52,6.81,7.55,4.89,5.48,7.74,5.10,8.17,7.67,7.07,5.80,6.10,7.15,7.88,9.06,6.85,4.88,6.74,8.76,8.53,6.72,7.21,7.42,8.29,8.56,9.25,6.63,7.49,6.67,6.79,5.19,8.20,7.97,8.64,7.36,6.72,5.90,5.53,6.44,7.35,5.18,8.25,5.68,6.29,6.69,6.08,7.42,7.10,7.14,7.10,6.60,6.35,5.99,6.17,9.05,6.01,7.77,6.27,5.81,7.80,9.89,4.39,6.83,6.53,8.15,6.68,6.87,6.31,6.83)
%\end{lstlisting}

\chapter{Теоретические сведения}

\section{Формулы для вычисления величин}

Минимальное и максимальное значения выборки, соответственно:
\begin{equation}
    \begin{aligned}
        M_{\max} = X_{(n)}\\
        M_{\min} = X_{(1)}
    \end{aligned}
\end{equation}

Размах выборки:
\begin{equation}
    R = M_{\max} - M_{\min}.
\end{equation}

Оценки математического ожидания и дисперсии, соответственно:
\begin{equation}
    \begin{aligned}
    \hat\mu(\vec X_n) &= \frac 1n \sum_{i=1}^n X_i\\
    S^2(\vec X_n) &= \frac 1{n-1} \sum_{i=1}^n (X_i-\overline X_n)^2
    \end{aligned}
\end{equation}

\section{Определение эмпирической плотности и гистограммы}

Пусть $\vec x=(x_1, ..., x_n)$ -- выборка из генеральной совокупности $X$, вектор $(x_{(1)}, ..., x_{(n)})$ -- вариационный ряд, построенный по этой выборке.  Если объем $n$ выборки велик, то значения $x_i$ группируют в так называемый интервальный статистический ряд. Для этого отрезок $J = [x_{(1)}; x_{(n)}]$ разбивают на $m$ равновеликих промежутков:

\begin{equation*}
    J_i = [x_{(1)} + (i - 1) \cdot \Delta; x_{(1)} + i \cdot \Delta), i = \overline{1; m - 1}
\end{equation*}

\begin{equation*}
    J_{m} = [x_{(1)} + (m - 1) \cdot \Delta; x_{(n)}]
\end{equation*}

\begin{equation*}
    \Delta = \frac{|J|}{m} = \frac{x_{(n)} - x_{(1)}}{m}
\end{equation*}

\textbf{Определение}

Интервальным статистическим рядом, отвечающим выборке $\vec x$ называют таблицу вида

\begin{table}[htb]
    \centering
    \begin{tabular}{|c|c|c|c|c|}
        \hline
        $J_1$ & ... & $J_i$ & ... & $J_m$ \\
        \hline
        $n_1$ & ... & $n_i$ & ... & $n_m$ \\
        \hline
    \end{tabular}
\end{table}где $n_i$ -- количество элементов выборки $\vec x$, которые $\in J_i$.
\\
Для выбора m используют формулу $m=[\log_2n]+2$ или $m=[\log_2n]+1.$

Пусть для данной выборки $\vec x$ построен интервальный статистический ряд $(J_i, n_i), i = \overline{1; m}$

\textbf{Определение}

Эмпирической функцией плотности распределения, соответсвующей выборке $\vec x$, называют функцию
\begin{equation}
    \hat f_n(x) =
    \begin{cases}
        \frac{n_i}{n \Delta}, x \in J_i, i = \overline{1; m} \\
        0, \text{иначе.} \\
    \end{cases}
\end{equation}


\textbf{Определение}
График эмпирической функции плотности называют гистограммой.


\section{Определение эмпирической функции распределения}

Пусть $\vec x = (x_1, ..., x_n)$ -- выборка из генеральной совокупности $X$. Обозначим $n(t, \vec x)$ -- число компонент вектора $\vec x$, которые меньше, чем $t$.

\textbf{Определение}

Эмпирической функцией распределения, построенной по выборке $\vec x$, называют функцию $F_n: \mathbb{R} \to \mathbb{R}$, определенную правилом 

\begin{equation}
    F_n(t) = \frac{n(t, \vec x)}{n}
\end{equation}



\chapter{Результат работы}

\section{Текст программы}

\lstinputlisting[style=mstyle]{../main.m}

\section{Результаты расчётов}

\begin{equation*}
    M_{\min} = -4,33 \\
\end{equation*}
\begin{equation*}
    M_{\max} = 0,01 \\
\end{equation*}
\begin{equation*}
    R = 4,34 \\
\end{equation*}
\begin{equation*}
    \hat\mu(\vec x_n) = 2,0585 \\
\end{equation*}
\begin{equation*}
    S^2(\vec x_n) = 0,994 \\
\end{equation*}
\begin{equation*}
    m = 8
\end{equation*}

\imgw{130mm}{f1}{Гистограмма и график функции плотности распределения вероятностей нормальной случайной величины с выборочными мат. ожиданием и дисперсией}

\imgw{130mm}{f2}{График эмперической функции распределения и функции распределения нормальной случайной величины с выборочными мат. ожиданием и дисперсией}
